\chapter{Introduction}

Unmanned Aerial Vehicles (UAVs) have seen explosive growth in the past thirty years, performing a multitude of military and civilian tasks including surveillance, reconnaissance, armed combat operations, search and rescue, forest fire management, and domestic policing \cite{sarris2001survey, valavanis2007advances}. A class of modern UAVs which have recently grown in popularity are quadrotors -  Vertical Take Off and Landing (VTOL) vehicles powered by four rotors arranged in a cross configuration. The main advantage of the quadrotor lies in its mechanical simplicity. Adjusting the speed of one or more of the vehicle's fixed-pitch rotors provides full attitude control, eliminating the need for the swash plate mechanism found on single rotor helicopters \cite{bramwell2001bramwell, gupte2012survey}. In spite of its mechanical simplicity, the quadrotor exhibits somewhat complex dynamics that are best modeled as a Multi-Input Multi-Output (MIMO) system.

Advances in MEMS sensors and light-weight high-powered lithium polymer batteries have contributed to the recent popularity of quadrotors, making them an attractive choice for research applications in flight dynamics and control, as in \cite{hoffmann2007quadrotor, kivrak2006design, mellinger2010control, michael2010grasp}. One problem of particular interest is the development of mathematical models representing system dynamics based on experimentally gathered data. System identification provides a mechanism to relate this input-output data to the underlying system dynamics. Traditionally, system identification techniques have focused on developing a system model which minimizes prediction error. Identification methods of this form are commonly known as Prediction Error Methods (PEMs). PEMs have seen widespread use in both theoretical and real-world applications, but experience difficulties with MIMO systems as noted in \cite{qin2006overview, viberg1995subspace}. Subspace identification methods have recently grown in popularity and offer an alternative approach to the identification problem. These methods have a foundation in linear algebra and overcome some of the issues found in PEMs when identifying MIMO systems \cite{katayama2005subspace}. While traditional subspace algorithms provide reliable results when identifying systems operating in open loop, modifications must be made to identify systems operating in closed loop to eliminate a bias introduced by feedback control. the presence of feedback control. It is the goal of this research project to apply subspace identification techniques using innovation estimation to a quadrotor using experimentally gathered closed-loop input and output data.


\section{Related Work}
Developing accurate dynamical models of quadrotors for simulation plays an important role in their development, test, and continuing operational use. Quadrotors are dynamically unstable and their dynamics are highly nonlinear. Because they have only four independent inputs (motor speeds) to control six degrees of freedom (three translational and three rotational), quadrotors are underactuated systems. 

Developing system models suitable for simulation from first principles is not commonly done due to the complexity of the resulting models and the difficulty of determining numerous unknown system parameters. Several groups have developed simple quadrotor models by directly measuring or estimating system parameters \cite{bresciani2008modelling, domingues2009quadrotor, kivrak2006design, pounds2006modelling, schreier2012modeling}, but these models are typically used to select initial controller gains during control system design, and model simulation results are generally not given. 



PEM (ARX family: \cite{chamberlain2011system})
ARMAX \cite{schreurs2013open}
\cite{miller2011open}
\cite{lee2011attitude}

N4SID in MATLAB (open loop) \cite{batmazdesign}





\section{Motivation and Contributions}



