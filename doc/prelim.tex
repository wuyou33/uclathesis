%%%%%%%%%%%%%%%%%%%%%%%%%%%%%%%%%%%%%%%%%%%%%%%%%%%%%%%%%%%%%%%%%%%%%%%%
%                                                                      %
%                          PRELIMINARY PAGES                           %
%                                                                      %
%%%%%%%%%%%%%%%%%%%%%%%%%%%%%%%%%%%%%%%%%%%%%%%%%%%%%%%%%%%%%%%%%%%%%%%%

\title          {Closed-Loop Subspace Identification of a Quadrotor}
\author         {Andrew G\@. Kee}
% Note: department is really your area of research.  I.e. leave out 'Department of'.
\department     {Engineering}
% Note:  degreeyear should be optional, but as of  5-Feb-96
% it seems required or you get a year of ``2''.   -johnh
\degreeyear     {2013}

%%%%%%%%%%%%%%%%%%%%%%%%%%%%%%%%%%%%%%%%%%%%%%%%%%%%%%%%%%%%%%%%%%%%%%%%

\chair          {Steve Gibson}
%\member         {Co-chair 3 name}
%\member         {Co-chair 2 name}
%\member         {Co-chair 1 name}

%%%%%%%%%%%%%%%%%%%%%%%%%%%%%%%%%%%%%%%%%%%%%%%%%%%%%%%%%%%%%%%%%%%%%%%%

%\dedication     {\textsl{To my mother \ldots \\
%                who---among so many other things--- \\
%                saw to it that I learned to touch-type \\
%                while I was still in elementary school}}

%%%%%%%%%%%%%%%%%%%%%%%%%%%%%%%%%%%%%%%%%%%%%%%%%%%%%%%%%%%%%%%%%%%%%%%%

%\acknowledgments {(Acknowledgments omitted for brevity.)}

%%%%%%%%%%%%%%%%%%%%%%%%%%%%%%%%%%%%%%%%%%%%%%%%%%%%%%%%%%%%%%%%%%%%%%%%

\abstract {
As quadrotors begin to see widespread use in military and civilian applications, accurate dynamical system models continue to play an important role in platform development, test, and operations.  System identification provides an approach to estimating dynamical system models from system input and output data. Subspace system identification methods identify models in their state space form by exploiting the structure of the state space representation. Common subspace identification methods provide reliable results when identifying systems operating in open loop, but produce biased results when identifying systems operating in the presence of feedback (i.e.\ closed-loop systems). The Innovation Estimation Method (IEM) is a subspace identification technique that provides an approach to eliminating this bias by pre-estimating the unknown innovation sequence before carrying out subspace identification. We describe the identification of an off-the-shelf quadrotor using experimentally gathered closed-loop input-output data through subspace identification with innovation estimation. We present experimental results illustrating the ability of a model identified via IEM to accurately represent  system dynamics in the presence of feedback control.
}
%%%%%%%%%%%%%%%%%%%%%%%%%%%%%%%%%%%%%%%%%%%%%%%%%%%%%%%%%%%%%%%%%%%%%%%%

\nomenclature{

\begin{longtable}{lll}
$\mathbb{R}$ 		&& Set of all reals\\
$\mathbb{Z}$ 		&& Set of all integers\\
$A$					&& System matrix\\
$B$					&& Input matrix\\
$C$					&& Output matrix\\
$D$					&& Feedforward matrix\\
$u$					&& System input vector\\
$y$					&& System output vector\\
$e$					&& System innovation vector\\
$x$					&& System state vector\\
$v$					&& Process noise vector\\
$w$					&& Measurement noise vector\\
$n$					&& Number of system states\\
$m$					&& Number of system inputs\\
$l$					&& Number of system outputs\\
$l$					&& Number of system outputs\\
$\sigma$			&& Singular value\\
$K$					&& Kalman filter gain\\
$U_k$				&& Block Hankel matrix of system inputs\\
$U_p$				&& Block Hankel matrix of past system inputs\\
$U_f$				&& Block Hankel matrix of future system inputs\\
$Y_k$				&& Block Hankel matrix of system outputs\\
$Y_p$				&& Block Hankel matrix of past system outputs\\
$Y_f$				&& Block Hankel matrix of future system outputs\\
$X_k$				&& Block Hankel matrix of system states\\
$E_k$				&& Block Hankel matrix of system innovation\\
$E_f$				&& Block Hankel matrix of future system innovation\\
$\Gamma_k$			&& Extended observability matrix\\
$\hat{\Gamma}_f$	&& Estimate of extended observability matrix over future horizon\\
$G_f$				&& Toeplitz matrix of future Markov parameters of stochastic subsystem\\
$H_f$				&& Toeplitz matrix of future Markov parameters of deterministic subsystem\\
$f$					&& Future time horizon\\
$p$					&& Past time horizon\\
$\Pi_{U_f}^\perp$ 	&& Orthogonal projection onto the column space of $U_f$\\
$I$					&& Identity matrix\\
$Z$					&& Instrumental variable matrix\\
$Z_p$				&& Instrumental variable matrix constructed of past input-output data\\
$\phi$				&& Roll\\
$\theta$			&& Pitch\\
$\dot\psi$				&& Yaw rate\\
\end{longtable}

}

%%%%%%%%%%%%%%%%%%%%%%%%%%%%%%%%%%%%%%%%%%%%%%%%%%%%%%%%%%%%%%%%%%%%%%%%
