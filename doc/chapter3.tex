\chapter{Subspace Identification Methods}
Subspace identification methods (SIM) provide an approach to identifing an unknown LTI system in its state space form using input/output data. Subspace methods rely on linear algebra techniques to extract the system matrices from the column space of the system's extended observability matrix. Considering the combined deterministic-stochastic LTI system introduced in the previous chapter, the subspace identification problem is: given a set of input and output data, estimate the system matrices $A$, $B$, $C$, and $D$ and the Kalman filter gain $K$ up to within a similarity transform. 




\subsection{Linear Algebra Tools}

\subsubsection*{Hankel Matrices}
A Hankel matrix is a matrix $H \in \mathbb{R}^{m\times n}$ with constant skew-diagonals. In other words, the value of the $(i, j)^{\mbox{th}}$ entry of $H$ depends only on the sum $i + j$.
\begin{equation*}
H_{m,n} = \begin{bmatrix}
h_1 & h_2 & \cdots & h_n\\
h_2 & h_3 & \cdots & h_{n+1}\\
\vdots & \vdots & \ddots & \vdots\\
h_m & h_{m+1} & \cdots & h_{m+n-1}
\end{bmatrix}
\end{equation*}
If each entry in the matrix is also a matrix, it is called a block Hankel matrix.


\subsubsection*{Fundamental Matrix Subspaces}
We require two of the fundamental matrix subspaces: the column space and the row space. The column space of a matrix $A \in \mathbb{R}^{m\times n}$ is the set of all linear combinations of the column vectors of $A$. The dimension of the column space is called the rank. The row space of a matrix $A \in \mathbb{R}^{m\times n}$ is the set of all linear combinations of the row vectors of $A$.


\subsubsection*{Projections}


\subsubsection*{Singular Value Decomposition}
Any matrix $A \in \mathbb{R}^{m\times n}$ can be decomposed by a singular value decomposition (SVD) given by
\begin{equation*}
A = U\Sigma V^T
\end{equation*}
where $U \in \mathbb{R}^{m\times m}$ and $V \in \mathbb{R}^{n\times n}$ are orthogonal matrices and $\Sigma \in \mathbb{R}^{m\times n}$ is diagonal matrix of the singular values of $A$ ordered such that
\begin{equation*}
\sigma_1 \geq \sigma_2 \geq \cdots \geq \sigma_k > 0
\end{equation*}




\section{Subspace Identification of Combined Deterministic-Stochastic Systems}
Considering the combined deterministic-stochastic LTI system in innovation form
\begin{subequations}
\begin{equation}x_{k+1} = Ax_k + Bu_k + Ke_k\end{equation}
\begin{equation}y_k = Cx_k + Du_k + e_k\end{equation}
\end{subequations}



A detailed derivation of this procedure follows


\section{Closed-Loop Subspace Identification}

\subsection{Identifying Systems Operating Under Feedback Control}

\subsection{Innovation Estimation Method}

\subsection{Whitening Filter}