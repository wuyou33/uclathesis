\documentclass[t]{beamer}   


% Setting Logo and Template
\usetheme{default}   
\usecolortheme{UCLAsquirrel}
\logo{\includegraphics[height=0.7cm]{logo_ucla_cw.png}}



% ======================================================================

\begin{document}

\title{Presentation About CS Stuff}
\author[A1, A2, and A3]{Author 1\and Author 2 \and Author 3}
\institute{UCLA, Los Angeles}
\begin{frame}
\titlepage
\end{frame}



% ======================================================================
\section{First Section}
% ======================================================================



\begin{frame}\frametitle{A simple Frame}
Some regular text 


\ 

\structure{Structured text:
\begin{itemize}
\item The verifiers $V$ share a key with the prover $P$ located at $pos$. 
\item The key is secret: any user
located at a position other than $pos$, has at most negligible info about the key.
\end{itemize}
}

\ 

Alerted text: 
\alert{Position-Verification (PV) Scheme}


\end{frame}

% ----------------------------------------------------------------------------------------------------------------


% ======================================================================
\section{Second Section, with subsections}
% ======================================================================
\subsection{The first Subsection}
% ----------------------------------------------------------------------------------------------------------------

\begin{frame}\frametitle{Theorem and Proof}

\begin{theorem}
Any presentation that uses a great color-theme, just works better.
\end{theorem}


\begin{proof}
I have a very nice proof, but this proof box is too small for it.
\end{proof}

%\begin{block}{block}
%Beamer's block box
%\end{block}

\begin{example}
Beamer's example environment
\end{example}

\begin{alertblock}{Alert}
Beamer's alertblock environment
\end{alertblock}

\end{frame}




% ----------------------------------------------------------------------------------------------------------------
\subsection{The Second and last Subsection}
% ----------------------------------------------------------------------------------------------------------------
\begin{frame}\frametitle{Lists example}
Some nested lists:
\begin{itemize}
\item First is just `itemize'
  \begin{itemize}
  \item a nested itemize
  \end{itemize}
\item Now lets play with numbers
\item Using enumerations
\begin{enumerate}
\item one
\item two
\item three
\end{enumerate}
\item That's it.
\end{itemize}

\end{frame}
% ======================================================================


\end{document}